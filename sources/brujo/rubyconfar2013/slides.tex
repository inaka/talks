\documentclass[utf8,hyperref={colorlinks=true}]{beamer}
\definecolor{links}{HTML}{2A1B81}
\hypersetup{colorlinks,linkcolor=,urlcolor=links}
\mode<presentation>
\usepackage{listings}
\usepackage{helvet}
\usetheme{Warsaw}
\usecolortheme{whale}
\usefonttheme[onlylarge]{structuresmallcapsserif}
\usefonttheme[onlysmall]{structurebold}
\usepackage{amsthm} % pushQED, popQED
\newenvironment{aquote}[1]{%
  \pushQED{#1}%
  \begin{quote}
}{%
  \par\noindent\hfill(\popQED)%
  \end{quote}%
}

\setbeamercovered{dynamic}
\setbeameroption{show notes}

\begin{document}
\title{Tu primer servidor en Erlang con SSE}
\subtitle{utilizando Cowboy y Sumo DB}
\author{Fernando Benavides (\textit{@elbrujohalcon})}
\institute{Inaka Labs}
\date{\today}
\logo{\includegraphics[height=0.5cm]{img/inaka-logo.png}}

\newcommand*\oldmacro{}%
\let\oldmacro\insertshorttitle%
\renewcommand*\insertshorttitle{%
  \oldmacro\hfill%
  \insertframenumber\,/\,\inserttotalframenumber}

%%%%%%%%%%%%%%%%%%%%%%%%%%%%%%%%%%%%%%%%%%%%%%%%%%%%%%%%%%%%%%%%%%%%%%
%% CODE SNIPPETS
%%%%%%%%%%%%%%%%%%%%%%%%%%%%%%%%%%%%%%%%%%%%%%%%%%%%%%%%%%%%%%%%%%%%%%
\definecolor{darkblue}{rgb}{0,0.08,0.45} 

\lstset{% general command to set parameter(s)
		mathescape=true,
		language=erlang,
		basicstyle=\ttfamily\small,
		keywordstyle=\color{blue}\bfseries,
		identifierstyle=\color{darkblue},
		stringstyle=\ttfamily,
		showstringspaces=false}

\defverbatim[colored]\startchild{%
\begin{lstlisting}[frame=single]
start_user(User) ->
  Manager =
    list_to_atom(
      "user-manager-" ++
        integer_to_list($\colorbox{yellow}{\texttt{random:uniform(?MANAGERS)}}$)),
  supervisor:start_child(Manager, [User]).
\end{lstlisting}
}
\defverbatim[colored]\startchildinit{%
\begin{lstlisting}[frame=single]
init([]) ->
  _ = random:seed(erlang:now()),
  Managers =
    [{list_to_atom("user-manager-" ++
                     integer_to_list(I)),
      $\colorbox{yellow}{\texttt{\{user\_mgr, start\_link, [I]\}}}$,
      permanent, brutal_kill, $\colorbox{yellow}{\texttt{supervisor}}$,
      [user_mgr]}
     || I <- $\colorbox{yellow}{\texttt{lists:seq(1, ?MANAGERS)}}$],
  {ok, {{one_for_one, 5, 10}, Managers}}.
\end{lstlisting}
}

\defverbatim[colored]\backlog{%
\begin{lstlisting}[frame=single]
gen_tcp:listen(Port,
  [binary, {packet, line}, {keepalive, true},
   {active, false}, {reuseaddr, true},
   $\colorbox{yellow}{\texttt{\{backlog, 128000\}}}$, {send_timeout, 32000},
   {send_timeout_close, true}]).
\end{lstlisting}
}
\defverbatim[colored]\backlogweb{%
\begin{lstlisting}[frame=single]
mochiweb_http:start(
  [{name, ?MODULE}, {loop, {?MODULE, loop}},
   $\colorbox{yellow}{\texttt{\{backlog, 128000\}}}$, {port, Port}]).
\end{lstlisting}
}
\defverbatim[colored]\db{%
\begin{lstlisting}[frame=single]
-define(REDIS_CONNECTIONS, 200).
-record(state, {redis :: $\colorbox{yellow}{\texttt{[pid()]}}$}).
$\ldots$
init([]) ->
  $\ldots$
  Redis =
    lists:map(
      fun(_) ->
        {ok, Conn} = erldis_client:start_link()
        Conn
      end, $\colorbox{yellow}{\texttt{lists:seq(1, ?REDIS\_CONNECTIONS)}}$),
  {ok, #state{redis = Redis}}.
\end{lstlisting}
}

\defverbatim[colored]\dbcall{%
\begin{lstlisting}[frame=single]
handle_call(Request, From, State) ->
  $\colorbox{yellow}{\texttt{[RedisConn|Redis] = State\#state.redis}}$,
  proc_lib:spawn_link(
    fun() ->
      Res = handle_call(Request, RedisConn),
      gen_server:reply(From, Res)
    end),
  {noreply, State#state{redis =
                          $\colorbox{yellow}{\texttt{Redis ++ [RedisConn]}}$}}.
\end{lstlisting}
}

\defverbatim[colored]\listener{%
\begin{lstlisting}[frame=single]
init([]) ->
  $\ldots$
  Listeners =
    [{list_to_atom("client-listener-" ++
                     integer_to_list(I)),
      $\colorbox{yellow}{\texttt{\{client\_listener, start\_link, [\textbf{I}]\}}}$,
      permanent, brutal_kill, worker,
      [client_listener]}
     || I <- $\colorbox{yellow}{\texttt{lists:seq(MinPort, MaxPort)}}$],
  {ok, {{one_for_one, 5, 10}, Listeners}}.
\end{lstlisting}
}

\defverbatim[colored]\suphandler{%
\begin{lstlisting}[frame=single]
EvtMgr =
  match_stream_match:event_manager(MatchId),
ok =
  gen_event:$\colorbox{yellow}{\texttt{add\_handler}}$(EvtMgr,
    {?MODULE, {MatchId,UserId,Client}}, self()),
$\colorbox{yellow}{\texttt{MgrRef = erlang:monitor(process, EvtMgr)}}$,
ClientRef = erlang:monitor(process, Client),
{reply, ok,
 State#state{matches =
  [{Client, MatchId, ClientRef, $\colorbox{yellow}{\texttt{MgrRef}}$}
   | State#state.matches]}}
\end{lstlisting}
}
\defverbatim[colored]\suphandlerinfo{%
\begin{lstlisting}[frame=single]
handle_info({$\colorbox{yellow}{\texttt{'DOWN'}}$,$\colorbox{yellow}{\texttt{Ref}}$,_,Client,_}, State) ->
  $\ldots$
  case $\colorbox{yellow}{\texttt{lists:keytake(Ref, 4, State\#state.matches)}}$ of
    {value, {Client,_,CRef,Ref}, OtherMatches} ->
      $\ldots$
\end{lstlisting}
}

\defverbatim[colored]\repeater{%
\begin{lstlisting}[frame=single]
start_link(Name, Source) ->
  {ok, Pid} = gen_event:start_link(Name),
  ok = gen_event:add_handler(
         Source, {?MODULE, Pid}, Pid),
  {ok, Pid}.
$\ldots$
init(Repeater) ->
  {ok, #state{mgr = Repeater}}.
$\ldots$
handle_event(Event, State) ->
  gen_event:notify(State#state.mgr, Event),
  {ok, State}.
\end{lstlisting}
}

\defverbatim[colored]\reply{%
\begin{lstlisting}[frame=single]
handle_call(Request, From, State) ->
  [RedisConn|Redis] = State#state.redis,
  proc_lib:spawn_link(
    fun() ->
      Res = handle_call(Request, RedisConn),
      $\colorbox{yellow}{\texttt{gen\_server:reply(From, Res)}}$
    end),
  {$\colorbox{yellow}{\texttt{noreply}}$, State#state{redis =
                          Redis ++ [RedisConn]}}.
\end{lstlisting}
}

\defverbatim[colored]\hibernate{%
\begin{lstlisting}[frame=single]
handle_cast(Event, State) ->
  $\ldots$
  {noreply, State, $\colorbox{yellow}{\texttt{hibernate}}$}.

$\ldots$

handle_call(Request, _From, State) ->
  $\ldots$
  {reply, Reply, State, $\colorbox{yellow}{\texttt{hibernate}}$}.
\end{lstlisting}
}

\defverbatim[colored]\init{%
\begin{lstlisting}[frame=single]
init(UserId) ->
  {ok, #state{user = UserId}, $\colorbox{yellow}{\texttt{0}}$}.
  
$\ldots$

handle_info($\colorbox{yellow}{\texttt{timeout}}$, State) ->
  case match_stream_db:user(State#state.user) of
  $\ldots$
\end{lstlisting}
}

\defverbatim[colored]\dispatcher{%
\begin{lstlisting}[frame=single]
event_manager(MatchId) ->
  $\colorbox{yellow}{\texttt{binary\_to\_atom(<<"event\_mgr@", MatchId/binary>>, utf8)}}$.

init(MatchId) ->
  {ok, EventMgr} =
    gen_event:start_link(
      $\colorbox{yellow}{\texttt{\{local, event\_manager(MatchId)\}}}$),
  $\ldots$
\end{lstlisting}
}

\defverbatim[colored]\tcpsample{%
\begin{lstlisting}[language=bash,basicstyle=\ttfamily\scriptsize]
> telnet <server> <port>    |2011-09-13 13:48:51: goal:
...                         |  player: Luna (7)
Welcome to Match Stream.    |  team: <<"tig">>
...                         |
V:2:elbrujohalcon           |2011-09-13 13:49:03: penalty:
2011-09-13 13:48:48: status:|  player: Martinez (6)
  home: <<"elp">>           |  team: <<"tig">>
  home_players:             |
     Albil (25)             |2011-09-13 13:49:04: card:
     ...                    |  player: Albil (25)
  home_score: 0             |  card: red
  visit: <<"tig">>          |  team: <<"elp">>
  visit_players:            |
     ..                     |2011-09-13 13:49:05: substitution:
  visit_score: 0            |  player_out: Fernandez (18)
  period: first             |  team: <<"elp">>
                            |  player_in: Silva (21)
\end{lstlisting}
}

%%%%%%%%%%%%%%%%%%%%%%%%%%%%%%%%%%%%%%%%%%%%%%%%%%%%%%%%%%%%%%%%%%%%%%

\frame{\titlepage}

\begin{frame}{Hello World!}
	\begin{itemize}
		\item<1> Soy programador desde que ten\'ia 10 a\~nos
		\item<2> Hago programaci\'on \emph{funcional} desde hace 5 a\~nos, en \emph{Erlang}
		\item<3> Soy \emph{Director of Engineering} en \textbf{Inaka}
		\item<4> Me dedico a dise\~nar y, a veces, construir servidores
		\item<5> \textbf{No soy} un programador \emph{Ruby}
		\item<6> Qu\'e hago ac\'a? \(o\_O\)
	\end{itemize}
\end{frame}

\appendix

\begin{frame}
	\begin{center}
		{\Huge Muchas Gracias!}
	\end{center}
\end{frame}

\end{document}