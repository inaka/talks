\documentclass[utf8,hyperref={colorlinks=true}]{beamer}
\definecolor{links}{HTML}{2A1B81}
\hypersetup{colorlinks,linkcolor=,urlcolor=links}
\mode<presentation>
\usepackage{multicol}
\usepackage{listings}
\usepackage{helvet}
\usetheme{Warsaw}
\usecolortheme{whale}
\usefonttheme[onlylarge]{structuresmallcapsserif}
\usefonttheme[onlysmall]{structurebold}
\usepackage{amsthm} % pushQED, popQED
\newenvironment{aquote}[1]{%
  \pushQED{#1}%
  \begin{quote}
}{%
  \par\noindent\hfill(\popQED)%
  \end{quote}%
}

\setbeamercovered{dynamic}
\setbeameroption{show notes}

\begin{document}
\title{Tu primer servidor en Erlang con SSE}
\subtitle{utilizando Cowboy y Sumo DB}
\author{Fernando Benavides (\textit{@elbrujohalcon})}
\institute{Inaka Labs}
\date{\today}
\logo{\includegraphics[height=0.5cm]{img/inaka-logo.png}}

\newcommand*\oldmacro{}%
\let\oldmacro\insertshorttitle%
\renewcommand*\insertshorttitle{%
  \oldmacro\hfill%
  \insertframenumber\,/\,\inserttotalframenumber}

%%%%%%%%%%%%%%%%%%%%%%%%%%%%%%%%%%%%%%%%%%%%%%%%%%%%%%%%%%%%%%%%%%%%%%
%% CODE SNIPPETS
%%%%%%%%%%%%%%%%%%%%%%%%%%%%%%%%%%%%%%%%%%%%%%%%%%%%%%%%%%%%%%%%%%%%%%
\definecolor{darkblue}{rgb}{0,0.08,0.45} 

\lstset{% general command to set parameter(s)
		mathescape=true,
		language=erlang,
		basicstyle=\ttfamily\scriptsize,
		keywordstyle=\color{blue}\bfseries,
		identifierstyle=\color{darkblue},
		stringstyle=\ttfamily,
		showstringspaces=false}

\defverbatim[colored]\thename{% <-- the name
\begin{lstlisting}[frame=single]

%%TODO: Code

\end{lstlisting}
}

\defverbatim[colored]\postsample{%
\begin{lstlisting}[language=bash]
curl -vX POST http://localhost:4004/news \
-H"Content-Type:application/json" \
-d'{ "source": "RubyConf", 
     "content": "@elbrujohalcon muestra su sistema para ..." }'
\end{lstlisting}
}
\defverbatim[colored]\postsampleresp{%
\begin{lstlisting}[language=bash]
> POST /news HTTP/1.1
> User-Agent: curl/7.30.0
> Host: localhost:4004
> Accept: */*
> Content-Type:application/json
> Content-Length: 50
>
< HTTP/1.1 204 No Content
< connection: keep-alive
< server: Cowboy
< date: Fri, 08 Nov 2013 20:06:01 GMT
< content-length: 0
<
\end{lstlisting}
}

\defverbatim[colored]\getsample{%
\begin{lstlisting}[language=bash]

\end{lstlisting}
}

\defverbatim[colored]\getsample{%
\begin{lstlisting}[language=bash]
curl -vX GET http://localhost:4004/news
\end{lstlisting}
}
\defverbatim[colored]\getsampleresph{%
\begin{lstlisting}[language=bash]
> GET /news HTTP/1.1
> User-Agent: curl/7.30.0
> Host: localhost:4004
> Accept: */*
>
\end{lstlisting}
}
\defverbatim[colored]\getsampleresphh{%
\begin{lstlisting}[language=bash]
< HTTP/1.1 200 OK
< transfer-encoding: chunked
< connection: keep-alive
< server: Cowboy
< date: Thu, 07 Nov 2013 14:31:10 GMT
< content-type: text/event-stream
<
event: RubyConf
data: La charla de @elbrujohalcon esta por comenzar

\end{lstlisting}
}

\defverbatim[colored]\getsamplerespa{%
\begin{lstlisting}[language=bash]
event: RubyConf
data: @elbrujohalcon muestra su sistema para ...

\end{lstlisting}
}

\defverbatim[colored]\getsamplerespb{%
\begin{lstlisting}[language=bash]
event: RubyConf 
data: el publico observa esta diapositiva :P

\end{lstlisting}
}

%%%%%%%%%%%%%%%%%%%%%%%%%%%%%%%%%%%%%%%%%%%%%%%%%%%%%%%%%%%%%%%%%%%%%%

\frame{\titlepage}

\begin{frame}{Hello World!}
	\begin{itemize}
		\item<1> Soy programador desde que ten\'ia 10 a\~nos
		\item<2> Hago programaci\'on \emph{funcional} desde hace 5 a\~nos, en \emph{Erlang}
		\item<3> Soy \emph{Director of Engineering} en \textbf{Inaka}
		\item<4> Me dedico a dise\~nar y, a veces, construir servidores
		\item<5> \textbf{No soy} un programador \emph{Ruby}
		\item<6> Qu\'e hago ac\'a? \(o\_O\)
	\end{itemize}
\end{frame}

\section{Introducci\'on}
\subsection{Descripci\'on}
\begin{frame}{Resumen}

\begin{description}
	\item[Escenario]
		\begin{itemize}
			\item Servidor con API tipo REST
			\item Clientes necesitan actualizaciones en \emph{Real-Time}
		\end{itemize}
	\ \\
	\pause
	\item[Soluci\'on]
		\begin{itemize}
			\item Se puede resolver con \emph{Ruby}? \textbf{S\'i}
			\item Existen otras soluciones? \textbf{S\'i}
		\end{itemize}
	\ \\
	\pause
	\item[Erlang]
		\begin{itemize}
			\item Es un paradigma distinto, requiere aprendizaje
			\item Es \textbf{ideal} para este tipo de escenarios
		\end{itemize}
\end{description}

\end{frame}

\subsection{Alcance}
\begin{frame}{Contenidos}
\begin{multicols}{2}
	\alert{En esta charla}
		\begin{itemize} \itemsep1em
			\item<+-> SSE
			\item<+-> Erlang / OTP \emph{b\'asico}
			\item<+-> Sumo DB \emph{b\'asico}
			\item<+-> Cowboy \emph{b\'asico}
		\end{itemize}
\columnbreak
	\alert{Fuera de esta charla}
		\begin{itemize} \itemsep1em
			\item<+-> REST \emph{avanzado}
			\item<+-> Erlang / OTP \emph{avanzado}
			\item<+-> Sumo DB \emph{avanzado}
			\item<+-> Elixir
		\end{itemize}
\end{multicols}
\end{frame}

\subsection{Canillita}
\begin{frame}{La Aplicaci\'on}
\begin{multicols}{2}
\includegraphics[height=.8\textheight]{img/paperboy.jpg} \\
\columnbreak
	\alert{\textit{{\huge Canillita}}}
	\ \\
	\ \\
	Simple API con dos endpoints:
	\ \\
	\ \\
		\begin{description}
			\item[POST /news]\ \\ para publicar noticias
			\item[GET /news]\ \\ para recibir noticias
		\end{description}
\end{multicols}
\end{frame}

\begin{frame}{Ejemplos}
\alert{\texttt{POST /news}}
\postsample
\visible<2>\postsampleresp
\end{frame}

\begin{frame}{Ejemplos}
\alert{\texttt{GET /news}}
\only<1,2>\getsample
\only<1,2,3>{\visible<2,3>\getsampleresph}
\visible<2,3,4>\getsampleresphh
\visible<3,4>\getsamplerespa
\visible<4>\getsamplerespb
\end{frame}

\appendix

\begin{frame}
	\begin{center}
		{\Huge Muchas Gracias!}
	\end{center}
\end{frame}

\end{document}