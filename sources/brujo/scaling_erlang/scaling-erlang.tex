\documentclass[utf8,hyperref={colorlinks=true}]{beamer}
\definecolor{links}{HTML}{2A1B81}
\hypersetup{colorlinks,linkcolor=,urlcolor=links}
\mode<presentation>
\usepackage{listings}
\usepackage{helvet}
\usetheme{Warsaw}
\usecolortheme{whale}
\usefonttheme[onlylarge]{structuresmallcapsserif}
\usefonttheme[onlysmall]{structurebold}
\usepackage{amsthm} % pushQED, popQED
\newenvironment{aquote}[1]{%
  \pushQED{#1}%
  \begin{quote}
}{%
  \par\noindent\hfill(\popQED)%
  \end{quote}%
}

\setbeamercovered{dynamic}
\setbeameroption{show notes}

\begin{document}
\title{Scaling Erlang Web Applications}
\subtitle{100 to 100K users at one web server}
\author{Fernando Benavides (\textit{@elbrujohalcon})}
\institute{Inaka Labs}
\date{\today}
\logo{\includegraphics[height=0.5cm]{img/inaka_leaf_logo.png}}

\newcommand*\oldmacro{}%
\let\oldmacro\insertshorttitle%
\renewcommand*\insertshorttitle{%
  \oldmacro\hfill%
  \insertframenumber\,/\,\inserttotalframenumber}

%%%%%%%%%%%%%%%%%%%%%%%%%%%%%%%%%%%%%%%%%%%%%%%%%%%%%%%%%%%%%%%%%%%%%%
%% CODE SNIPPETS
%%%%%%%%%%%%%%%%%%%%%%%%%%%%%%%%%%%%%%%%%%%%%%%%%%%%%%%%%%%%%%%%%%%%%%
\definecolor{darkblue}{rgb}{0,0.08,0.45} 

\lstset{% general command to set parameter(s)
		mathescape=true,
		language=erlang,
		basicstyle=\ttfamily\small,
		keywordstyle=\color{blue}\bfseries,
		identifierstyle=\color{darkblue},
		stringstyle=\ttfamily,
		showstringspaces=false}

\defverbatim[colored]\startchild{%
\begin{lstlisting}[frame=single]
start_user(User) ->
  Manager =
    list_to_atom(
      "user-manager-" ++
        integer_to_list($\colorbox{yellow}{\texttt{random:uniform(?MANAGERS)}}$)),
  supervisor:start_child(Manager, [User]).
\end{lstlisting}
}
\defverbatim[colored]\startchildinit{%
\begin{lstlisting}[frame=single]
init([]) ->
  _ = random:seed(erlang:now()),
  Managers =
    [{list_to_atom("user-manager-" ++
                     integer_to_list(I)),
      $\colorbox{yellow}{\texttt{\{user\_mgr, start\_link, [I]\}}}$,
      permanent, brutal_kill, $\colorbox{yellow}{\texttt{supervisor}}$,
      [user_mgr]}
     || I <- $\colorbox{yellow}{\texttt{lists:seq(1, ?MANAGERS)}}$],
  {ok, {{one_for_one, 5, 10}, Managers}}.
\end{lstlisting}
}

\defverbatim[colored]\backlog{%
\begin{lstlisting}[frame=single]
gen_tcp:listen(Port,
  [binary, {packet, line}, {keepalive, true},
   {active, false}, {reuseaddr, true},
   $\colorbox{yellow}{\texttt{\{backlog, 128000\}}}$, {send_timeout, 32000},
   {send_timeout_close, true}]).
\end{lstlisting}
}
\defverbatim[colored]\backlogweb{%
\begin{lstlisting}[frame=single]
mochiweb_http:start(
  [{name, ?MODULE}, {loop, {?MODULE, loop}},
   $\colorbox{yellow}{\texttt{\{backlog, 128000\}}}$, {port, Port}]).
\end{lstlisting}
}
\defverbatim[colored]\db{%
\begin{lstlisting}[frame=single]
-define(REDIS_CONNECTIONS, 200).
-record(state, {redis :: $\colorbox{yellow}{\texttt{[pid()]}}$}).
$\ldots$
init([]) ->
  $\ldots$
  Redis =
    lists:map(
      fun(_) ->
        {ok, Conn} = erldis_client:start_link()
        Conn
      end, $\colorbox{yellow}{\texttt{lists:seq(1, ?REDIS\_CONNECTIONS)}}$),
  {ok, #state{redis = Redis}}.
\end{lstlisting}
}

\defverbatim[colored]\dbcall{%
\begin{lstlisting}[frame=single]
handle_call(Request, From, State) ->
  $\colorbox{yellow}{\texttt{[RedisConn|Redis] = State\#state.redis}}$,
  proc_lib:spawn_link(
    fun() ->
      Res = handle_call(Request, RedisConn),
      gen_server:reply(From, Res)
    end),
  {noreply, State#state{redis =
                          $\colorbox{yellow}{\texttt{Redis ++ [RedisConn]}}$}}.
\end{lstlisting}
}

\defverbatim[colored]\listener{%
\begin{lstlisting}[frame=single]
init([]) ->
  $\ldots$
  Listeners =
    [{list_to_atom("client-listener-" ++
                     integer_to_list(I)),
      $\colorbox{yellow}{\texttt{\{client\_listener, start\_link, [\textbf{I}]\}}}$,
      permanent, brutal_kill, worker,
      [client_listener]}
     || I <- $\colorbox{yellow}{\texttt{lists:seq(MinPort, MaxPort)}}$],
  {ok, {{one_for_one, 5, 10}, Listeners}}.
\end{lstlisting}
}

\defverbatim[colored]\suphandler{%
\begin{lstlisting}[frame=single]
EvtMgr =
  match_stream_match:event_manager(MatchId),
ok =
  gen_event:$\colorbox{yellow}{\texttt{add\_handler}}$(EvtMgr,
    {?MODULE, {MatchId,UserId,Client}}, self()),
$\colorbox{yellow}{\texttt{MgrRef = erlang:monitor(process, EvtMgr)}}$,
ClientRef = erlang:monitor(process, Client),
{reply, ok,
 State#state{matches =
  [{Client, MatchId, ClientRef, $\colorbox{yellow}{\texttt{MgrRef}}$}
   | State#state.matches]}}
\end{lstlisting}
}
\defverbatim[colored]\suphandlerinfo{%
\begin{lstlisting}[frame=single]
handle_info({$\colorbox{yellow}{\texttt{'DOWN'}}$,$\colorbox{yellow}{\texttt{Ref}}$,_,Client,_}, State) ->
  $\ldots$
  case $\colorbox{yellow}{\texttt{lists:keytake(Ref, 4, State\#state.matches)}}$ of
    {value, {Client,_,CRef,Ref}, OtherMatches} ->
      $\ldots$
\end{lstlisting}
}

\defverbatim[colored]\repeater{%
\begin{lstlisting}[frame=single]
start_link(Name, Source) ->
  {ok, Pid} = gen_event:start_link(Name),
  ok = gen_event:add_handler(
         Source, {?MODULE, Pid}, Pid),
  {ok, Pid}.
$\ldots$
init(Repeater) ->
  {ok, #state{mgr = Repeater}}.
$\ldots$
handle_event(Event, State) ->
  gen_event:notify(State#state.mgr, Event),
  {ok, State}.
\end{lstlisting}
}

\defverbatim[colored]\reply{%
\begin{lstlisting}[frame=single]
handle_call(Request, From, State) ->
  [RedisConn|Redis] = State#state.redis,
  proc_lib:spawn_link(
    fun() ->
      Res = handle_call(Request, RedisConn),
      $\colorbox{yellow}{\texttt{gen\_server:reply(From, Res)}}$
    end),
  {$\colorbox{yellow}{\texttt{noreply}}$, State#state{redis =
                          Redis ++ [RedisConn]}}.
\end{lstlisting}
}

\defverbatim[colored]\hibernate{%
\begin{lstlisting}[frame=single]
handle_cast(Event, State) ->
  $\ldots$
  {noreply, State, $\colorbox{yellow}{\texttt{hibernate}}$}.

$\ldots$

handle_call(Request, _From, State) ->
  $\ldots$
  {reply, Reply, State, $\colorbox{yellow}{\texttt{hibernate}}$}.
\end{lstlisting}
}

\defverbatim[colored]\init{%
\begin{lstlisting}[frame=single]
init(UserId) ->
  {ok, #state{user = UserId}, $\colorbox{yellow}{\texttt{0}}$}.
  
$\ldots$

handle_info($\colorbox{yellow}{\texttt{timeout}}$, State) ->
  case match_stream_db:user(State#state.user) of
  $\ldots$
\end{lstlisting}
}

\defverbatim[colored]\dispatcher{%
\begin{lstlisting}[frame=single]
event_manager(MatchId) ->
  $\colorbox{yellow}{\texttt{binary\_to\_atom(<<"event\_mgr@", MatchId/binary>>, utf8)}}$.

init(MatchId) ->
  {ok, EventMgr} =
    gen_event:start_link(
      $\colorbox{yellow}{\texttt{\{local, event\_manager(MatchId)\}}}$),
  $\ldots$
\end{lstlisting}
}

\defverbatim[colored]\tcpsample{%
\begin{lstlisting}[language=bash,basicstyle=\ttfamily\scriptsize]
> telnet <server> <port>    |2011-09-13 13:48:51: goal:
...                         |  player: Luna (7)
Welcome to Match Stream.    |  team: <<"tig">>
...                         |
V:2:elbrujohalcon           |2011-09-13 13:49:03: penalty:
2011-09-13 13:48:48: status:|  player: Martinez (6)
  home: <<"elp">>           |  team: <<"tig">>
  home_players:             |
     Albil (25)             |2011-09-13 13:49:04: card:
     ...                    |  player: Albil (25)
  home_score: 0             |  card: red
  visit: <<"tig">>          |  team: <<"elp">>
  visit_players:            |
     ..                     |2011-09-13 13:49:05: substitution:
  visit_score: 0            |  player_out: Fernandez (18)
  period: first             |  team: <<"elp">>
                            |  player_in: Silva (21)
\end{lstlisting}
}

%%%%%%%%%%%%%%%%%%%%%%%%%%%%%%%%%%%%%%%%%%%%%%%%%%%%%%%%%%%%%%%%%%%%%%

\frame{\titlepage} 

\begin{frame}{Hello World!}
	\begin{itemize}
		\item<+-> I'm a developer since I was 10
		\item<+-> I worked with Visual Basic, C\#, .NET, Javascript \ldots
		\item<+-> I switched to functional programming in 2008
		\item<+-> I wrote my thesis project in Haskell
		\item<+-> I'm an Erlang developer since then
	\end{itemize}
\end{frame}
\begin{frame}{Inaka}
	\begin{itemize}
		\item We're based in Argentina
		\item We do Erlang consulting
		\item We often build end-to-end applications with
			\begin{itemize}
				\item Erlang backends
				\item iOS clients
				\item Ruby admin tools
			\end{itemize}
	\end{itemize}
\end{frame}

\section{Introduction}
\subsection{Description}
\begin{frame}{Introduction}
	My talk is on the scalability of a \emph{web} project \pause that has an \emph{HTTP API} \pause and a component that keeps clients \emph{connected} to the server for \emph{long periods} of time.\\ ~ \\ \pause
	It's a design pattern seen in many places:
	\begin{itemize}
		\item Chat Rooms
		\item Social Sites
		\item Sport Sites
	\end{itemize}
\end{frame}

\subsection{Scope}
\begin{frame}{Scope}
	\emph{We will improve the way we use}
	\begin{itemize}
		\item OTP behaviours
		\item TCP and HTTP connections
		\item Underlaying system configurations
	\end{itemize}
	\pause
	\emph{We will \textbf{not} deal with}
	\begin{itemize}
		\item Multiple machines/nodes
		\item Database choices and/or implementations
	\end{itemize}
\end{frame}

\section{Match Stream}
\subsection{Idea}
\begin{frame}[t]{Match Stream}{General Idea}
	\only<1>{
		\emph{A soccer match is played at some stadium}
		\includegraphics[width=\textwidth]{img/overview-1.png}}
	\only<2>{
		\emph{Soccer fans are connected to the internet in their offices}
		\includegraphics[width=\textwidth]{img/overview-2.png}}
	\only<3>{
		\emph{A reporter is at the stadium with his device}
		\includegraphics[width=\textwidth]{img/overview-3.png}}
	\only<4>{
		\emph{\textsc{MatchStream} \textbf{connects} them in \textbf{real time}}
		\includegraphics[width=\textwidth]{img/overview-4.png}}
\end{frame}
\begin{frame}{Match Stream}{Requirements}
	\begin{itemize}
		\item<+-> Many concurrent users connecting at the same time
		\item<+-> Two-hour-long bursts of connections followed by long periods of inactivity
		\item<+-> Real-time updates
	\end{itemize}
\end{frame}

\subsection{Design}
\begin{frame}{Match Stream}{General Design}
	\includegraphics[width=\textwidth]{img/MatchStream.png}
\end{frame}
\begin{frame}{Match Stream}{Sample}
	\tcpsample
\end{frame}
\begin{frame}[t]{Match Stream}{Architecture}
	\begin{center}
		\includegraphics[width=\textwidth]{img/architecture-1.png}
	\end{center}
\end{frame}
\begin{frame}{Components}
	\begin{columns}
		\column{.5\textwidth}
			\only<1>{
				\begin{center}
					\includegraphics[width=\textwidth]{img/architecture-1-1.png}
				\end{center}}
			\only<2>{
				\begin{center}
					\includegraphics[width=\textwidth]{img/architecture-1-2.png}
				\end{center}}
			\only<3>{
				\begin{center}
					\includegraphics[width=.75\textwidth]{img/architecture-1-3.png}
				\end{center}}
			\only<4>{
				\begin{center}
					\includegraphics[width=\textwidth]{img/architecture-1-4.png}
				\end{center}}
			\only<5>{
				\begin{center}
					\includegraphics[width=\textwidth]{img/architecture-1-5.png}
				\end{center}}
			\only<6>{
				\begin{center}
					\includegraphics[width=\textwidth]{img/architecture-1-6.png}
				\end{center}}
		\column{.5\textwidth}
			\only<1>{
				\begin{description}
					\item[client\_listener]
						\texttt{gen\_server}. Listens on a TCP port to receive client connections
					\end{description}}
			\only<2>{
				\begin{description}
					\item[client\_sup]
						\texttt{supervisor}. Supervises connection processes
					\item[client]
						\texttt{gen\_fsm}. \\ Handles a TCP connection
				\end{description}}
			\only<3>{
				\begin{description}
					\item[web]
						\texttt{mochiweb server}. Listens for HTTP API calls
				\end{description}}
			\only<4>{
				\begin{description}
					\item[user\_sup]
						\texttt{supervisor}. Supervises user processes
					\item[user]
						\texttt{gen\_server}. Subscribes to match dispatchers and sends events to clients
				\end{description}}
			\only<5>{
				\begin{description}
					\item[match\_sup]
						\texttt{supervisor}. Supervises match processes
					\item[match]
						\texttt{gen\_server}. Listens to match events, stores them
					\item[dispatcher]
						\texttt{gen\_event dispatcher}. Delivers match events
				\end{description}}
			\only<6>{
				\begin{description}
					\item[db]
						\texttt{gen\_server}. Processes database operations
					\item[connection]
						\texttt{erldis client}. Handles the connection to the database
		\end{description}}
	\end{columns}
\end{frame}

\section{Scaling}
\begin{frame}{Lesson Learned}
	\begin{center}
		\huge \emph{Simply using Erlang to build your system is \textbf{not enough} to ensure \textbf{scalability}}
	\end{center}
\end{frame}

\begin{frame}{Measures}
	\begin{description}
		\item[N] \emph{Connections}. Number of connections the server can handle
		\item[C] \emph{Concurrency}. Number of connections starting at a time
		\item[ART] \emph{Average Response Time}. How much time it takes for the server to send an event
	\end{description}
\end{frame}
\begin{frame}{Tools}
	\begin{description}
		\item[Test Client]~\\ We create our own test client for TCP connections
		\item[ApacheBench]~\\ To test API calls
		\item[entop]~\\ We use it to check server processes
	\end{description}
\end{frame}

\subsection{Stage 0: Baseline}
\begin{frame}{Stage 0}{Establishing a Baseline}
	\textsc{Goals}
	\begin{itemize}
		\item Find how much the system can handle
	\end{itemize}
	\pause
	\textsc{Steps}
	\begin{itemize}
		\item Create automated testers
		\item Install and start the system on a \emph{clean} machine
		\item Run the tests on the server's local network
		\item Test repeatedly adjusting our parameters to maximize them
		\item Have a human using the system himself
	\end{itemize}
\end{frame}
\begin{frame}{Stage 1}{Results}
	\begin{columns}
		\column{.66\textwidth}
			\includegraphics[top=-1,width=\textwidth]{img/results-1.jpg}
		\column{.33\textwidth}
			\begin{description}
				\item[N] \textbf{\color{red}\Large 1000}
				\item[C] \textbf{\color{red}\Large 4}
				\item[ART] \textbf{\color{red}\Large 26s}
			\end{description}
	\end{columns}
\end{frame}

\subsection{Stage 1: OS Tune}
\begin{frame}{Stage 1}{Tune the OS and the VM}
	\textsc{Goals}
	\begin{itemize}
		\item Improve the underlying Operating System
		\item Improve the Erlang VM Configuration
	\end{itemize}
	\pause
	\textsc{Settings to Tune Up}
	\begin{itemize}
		\item Open files limit
		\item TCP connections limit
		\item TCP backlog size
		\item TCP memory allocation
		\item Number of Erlang processes
	\end{itemize}
\end{frame}
\begin{frame}{Stage 1}{Results}
	\begin{columns}
		\column{.66\textwidth}
			\includegraphics[top=-1,width=\textwidth]{img/results-2.jpg}
		\column{.33\textwidth}
			\begin{description}
				\item[N] \textbf{\colorbox{yellow}{\Large 4000}}
				\item[C] \textbf{\color{red}\Large 4}
				\item[ART] \textbf{\colorbox{yellow}{\color{red}\Large 31s}}
			\end{description}
	\end{columns}
\end{frame}

\subsection{Stage 2: Tune the Code}
\subsubsection{TCP Tuning}
\begin{frame}{Stage 2.1}{Connection Tweaks}
	\begin{description}
		\item<+->[Backlog]\ \\
			\begin{itemize}
				\item It defaults to 5, we should increase it
				\item Don't forget to TCP tune your HTTP server
			\end{itemize}
	\end{description}
	\begin{aquote}{Erlang Docs}The backlog value defines the maximum length that the queue of pending connections may grow to.\end{aquote}
\end{frame}
\begin{frame}{Stage 2.1}{Connection Tweaks}
	\begin{description}
		\item[client\_listener]~\\\backlog
		\item[web]~\\\backlogweb
	\end{description}
\end{frame}
\begin{frame}{Stage 2.1}{Connection Tweaks}
	\begin{description}
		\item<+->[Outbound Connections]\ \\
			\begin{itemize}
				\item e.g., database connections
				\item Don't use just one of them
				\item You may have separate connections for different purposes
			\end{itemize}
	\end{description}
\end{frame}
\begin{frame}{Stage 2.1}{Connection Tweaks}
\only<1>{\db}
\only<2>{\dbcall}
\end{frame}
\begin{frame}{Stage 2.1}{Connection Tweaks}
	\begin{description}
		\item<+->[Listeners]\ \\
			\begin{itemize}
				\item You can listen to more than one port
				\item For unified urls, use \emph{nginx} in front of the server
			\end{itemize}
	\end{description}
\end{frame}
\begin{frame}{Stage 2.1}{Connection Tweaks}
\listener
\end{frame}
\begin{frame}{Stage 2.1}{Connection Tweaks}
	\begin{columns}
		\column{.5\textwidth}
			\begin{center}
				\includegraphics[height=.7\textheight]{img/architecture-2.png}
			\end{center}
		\column{.5\textwidth}
			\begin{description}
				\item[listener]
					\texttt{gen\_server}. Listens on a TCP port to receive client connections
				\item[web]
					\texttt{mochiweb server}. Listens for HTTP API calls on a particular port
				\item[connection]
					\texttt{erldis client}. Handles the connection to the database
			\end{description}
	\end{columns}
\end{frame}
\begin{frame}{Stage 2.1}{Results}
	\begin{columns}
		\column{.66\textwidth}
			\includegraphics[top=-1,width=\textwidth]{img/results-3-1.jpg}
		\column{.33\textwidth}
			\begin{description}
				\item[N] \textbf{\colorbox{yellow}{\Large 8000}}
				\item[C] \textbf{\colorbox{yellow}{\Large 500}}
				\item[ART] \textbf{\colorbox{yellow}{\color{red}\Large 15s}}
			\end{description}
	\end{columns}
\end{frame}

\subsubsection{OTP}
\begin{frame}{Stage 2.2}{gen\textunderscore event}
	\begin{description}
		\item<+->[sup\textunderscore handler]\ \\
			\begin{itemize}
				\item Don't use it
				\item It exponentially increases the number of \texttt{'EXIT'} messages sent to the subscribers because it notifies anybody of any process termination
				\item Monitor the processes instead
			\end{itemize}
	\end{description}
\end{frame}
\begin{frame}{Stage 2.2}{gen\textunderscore event}
	\only<1>{\suphandler}
	\only<2>{\suphandlerinfo}
\end{frame}
\begin{frame}{Stage 2.2}{gen\textunderscore event}
	\begin{description}
		\item<+->[Long Delivery Queues]\ \\
			\begin{itemize}
				\item The more subscribers a gen\_event has, the longer it takes for it to process a notification
				\item Distribute the work
				\item Use gen\_event \emph{repeaters}
			\end{itemize}
	\end{description}
\end{frame}
\begin{frame}{Stage 2.2}{gen\textunderscore event}
\repeater
\end{frame}
\begin{frame}{Stage 2.2}{gen\textunderscore event}
	\begin{columns}
		\column{.5\textwidth}
			\begin{center}
				\includegraphics[height=.75\textheight]{img/architecture-3-2.png}
			\end{center}
		\column{.5\textwidth}
			\begin{description}
				\item[repeater]
					\texttt{gen\_event dispatcher}. It's subscribed to \texttt{dispatcher} and \emph{repeats} the received events to its subscribers
			\end{description}
	\end{columns}
\end{frame}
\begin{frame}{Stage 2.2}{Results}
	\begin{columns}
		\column{.66\textwidth}
			\includegraphics[top=-1,width=\textwidth]{img/results-3-2.jpg}
		\column{.33\textwidth}
			\begin{itemize}
				\item[N] \textbf{\Large 8000}
				\item[C] \textbf{\colorbox{yellow}{\Large 1000}}
				\item[ART] \textbf{\colorbox{yellow}{\Large 2s}}
			\end{itemize}
	\end{columns}
\end{frame}
\begin{frame}{Stage 2.3}{gen\textunderscore server}
	\begin{description}
		\item<+->[Queued Calls]\ \\
			Remember \texttt{gen\textunderscore server:reply/2}
	\end{description}
\end{frame}
\begin{frame}{Stage 2.3}{gen\textunderscore server}
\reply
\end{frame}
\begin{frame}{Stage 2.3}{gen\textunderscore server}
	\begin{description}
		\item<+->[Memory Footprint]\ \\
			Remember \texttt{hibernate}~\\
	\end{description}
	\begin{aquote}{Erlang Docs}
		Puts the calling process into a wait state where its memory allocation has been reduced as much as possible, which is useful if the process does not expect to receive any messages in the near future.
	\end{aquote}
\end{frame}
\begin{frame}{Stage 2.3}{gen\textunderscore server}
\hibernate
\end{frame}
\begin{frame}{Stage 2.3}{gen\textunderscore server}
	\begin{description}
		\item<+->[Long startup time]\ \\
			\begin{itemize}
				\item Initialize your gen\textunderscore servers in a $0$ timeout
				\item Move initialization code to \texttt{handle\_info}
			\end{itemize}
	\end{description}
\end{frame}
\begin{frame}{Stage 2.3}{gen\textunderscore server}
\init
\end{frame}
\begin{frame}{Stage 2.3}{Results}
	\begin{columns}
		\column{.64\textwidth}
			\includegraphics[top=-1,width=\textwidth]{img/results-3-3.jpg}
		\column{.36\textwidth}
			\begin{description}
				\item[N] \textbf{\colorbox{yellow}{\Large 40K}}
				\item[C] \textbf{\colorbox{yellow}{\Large 5K}}
				\item[ART] \textbf{\colorbox{yellow}{\Large 25ms}}
			\end{description}
	\end{columns}
\end{frame}
\begin{frame}{Stage 2.4}{supervisors}
	\begin{description}
		\item<+->[Simple One for One Supervisors]\ \\
			\begin{itemize}
				\item Sometimes \texttt{simple\textunderscore one\textunderscore for\textunderscore one} supervisors get \alert{overburdened} because they have too many children
				\item Use a supervisor hierarchy
			\end{itemize}
	\end{description}
\end{frame}
\begin{frame}{Stage 2.4}{supervisors}
\only<1>{\startchildinit}
\only<2>{\startchild}
\end{frame}
\begin{frame}{Stage 2.4}{supervisors}
	\begin{columns}
		\column{.5\textwidth}
			\begin{center}
				\includegraphics[height=.65\textheight]{img/architecture-3-4.png}
			\end{center}
		\column{.5\textwidth}
			\begin{description}
				\item[user\_mgr]
					\texttt{supervisor}. Supervises a group of users processes
				\item[client\_mgr]
					\texttt{supervisor}. Supervises a group of connection processes
			\end{description}
	\end{columns}
\end{frame}
\begin{frame}{Stage 2.4}{Results}
	\begin{columns}
		\column{.66\textwidth}
			\includegraphics[top=-1,width=\textwidth]{img/results-3-4.jpg}
		\column{.33\textwidth}
			\begin{description}
				\item[N] \textbf{\colorbox{yellow}{\Large 50K}}
				\item[C] \textbf{\colorbox{yellow}{\Large 8K}}
				\item[ART] \textbf{\Large 25ms}
			\end{description}
	\end{columns}
\end{frame}

\subsubsection{Other Stuff}
\begin{frame}{Stage 2.5}{Other Processes}
	\begin{description}
		\item<+->[Logging]\ \\
			\begin{itemize}
				\item Use a good logging system
				\item Choose carefully what to log
			\end{itemize}
	\end{description}
\end{frame}
\begin{frame}{Stage 2.5}{Other Processes}
	\begin{description}
		\item<+->[Registration]\ \\
			\begin{itemize}
				\item If everybody agrees on a name, nobody has to find where it is
				\item You can always register processes \alert{both} locally and globally
			\end{itemize}
	\end{description}
\end{frame}
\begin{frame}{Stage 2.5}{Other Processes}
\dispatcher
\end{frame}
\begin{frame}{Stage 2.5}{Results}
	\begin{columns}
		\column{.64\textwidth}
			\includegraphics[top=-1,width=\textwidth]{img/results-3-5.jpg}
		\column{.35\textwidth}
			\begin{description}
				\item[N] \textbf{\Large 50K}
				\item[C] \textbf{\Large 8K}
				\item[ART] \textbf{\colorbox{yellow}{\Large 10ms}}
			\end{description}
	\end{columns}
\end{frame}

\subsection{Stage 3: Multi-Node Tuning}
\begin{frame}{Stage 3}{Adding Nodes}
	\textsc{Goals}
	\begin{itemize}
		\item Find the best system topology
	\end{itemize}
	\pause
	\textsc{Steps}
	\begin{itemize}
		\item Prepare the system to run in more than one node
		\item Decide if nodes should be connected or independent
		\item Decide if nodes should be on the same machine or not
		\item Determine which processes should be registered globally
	\end{itemize}
\end{frame}
\begin{frame}{Stage 3}{Adding Nodes}
	\begin{description}
		\item[For MatchStream]~\\
			\begin{itemize}
				\item We leave the \texttt{match} processes registered locally but we connect them using \texttt{pg2}
				\item We split \texttt{db} in two:
					\begin{itemize}
						\item an unique (and therefore) globally registered \texttt{db\_writer}
						\item a \texttt{db\_reader} per node
					\end{itemize}
			\end{itemize}
	\end{description}
\end{frame}
\begin{frame}{Stage 3}{Adding Nodes}
	\begin{columns}
		\column{.5\textwidth}
			\begin{center}
				\includegraphics[height=.65\textheight]{img/architecture-4.png}
			\end{center}
		\column{.5\textwidth}
			\begin{description}
				\item[db\_reader]
					\texttt{gen\_server}. One per node. Processes db read operations
				\item[db\_writer]
					\texttt{gen\_server}. One per \alert{system}. Processes db write operations
			\end{description}
	\end{columns}
\end{frame}
\begin{frame}{Stage 3}{Results}
	\begin{columns}
		\column{.65\textwidth}
			\includegraphics[top=-1,width=\textwidth]{img/results-4.jpg}
		\column{.34\textwidth}
			With four nodes in the same machine:
			\begin{description}
				\item[N] \textbf{\colorbox{yellow}{\Large 100K}}
				\item[C] \textbf{\colorbox{yellow}{\Large 32K}}
				\item[ART] \textbf{\Large 10ms}
			\end{description}
	\end{columns}
\end{frame}

\section{Final Words}
\subsection{Summary}
\begin{frame}{Summary}
	\begin{columns}
		\column<+->{.29\textwidth}
			With started with:
			\begin{description}
				\item[N] \textbf{\Large 1K}
				\item[C] \textbf{\Large 4}
				\item[ART] \textbf{\Large 26s}
			\end{description}
		\column<+->{.33\textwidth}
			We scale up to:
			\begin{description}
				\item[N] \textbf{\Large 100K}
				\item[C] \textbf{\Large 32K}
				\item[ART] \textbf{\Large 10ms}
			\end{description}
		\column<+->{.38\textwidth}
			Our improvements:
			\begin{description}
				\item[N] \textbf{\colorbox{yellow}{100x}}
				\item[C] \textbf{\colorbox{yellow}{8000x}}
				\item[ART] \textbf{\colorbox{yellow}{2600x}}
			\end{description}
	\end{columns}
\end{frame}
\begin{frame}{Summary}
	\begin{itemize}
		\item<+-> This is an \alert{iterative} process
		\item<+-> It proved itself useful in both experimental and real-life systems
		\item<+-> It gets improved with every system we scale
	\end{itemize}
\end{frame}
\subsection{Questions}
\begin{frame}{Questions}
	\begin{center}
		\includegraphics[width=\textwidth]{img/theriddler.jpg}
	\end{center}
\end{frame}
\begin{frame}{Resources}
	\begin{description}
		\item[About me]~\\
			\begin{itemize}
				\item I'm \textbf{\href{http://twitter.com/elbrujohalcon}{@elbrujohalcon}} at Twitter
				\item I'm \href{http://github.com/elbrujohalcon}{elbrujohalcon} at Github
			\end{itemize}
		\item[About Inaka]~\\
			\begin{itemize}
				\item Check our website (soon :P): \href{http://inakanetworks.com}{http://inaka.net}
				\item And our Blog that's on the same site
			\end{itemize}
		\item[About MatchStream]~\\
			\begin{itemize}
				\item The project is on Github: \href{http://github.com/inaka/match\_stream}{http://github.com/inaka/match\_stream}
				\item This talk, too: \href{http://github.com/inaka/erlang\_factory/brujo}{http://github.com/inaka/erlang\_factory/brujo}
			\end{itemize}
	\end{description}
\end{frame}

\appendix

\begin{frame}
	\begin{center}
		{\Huge Thanks!}
	\end{center}
\end{frame}

\end{document}